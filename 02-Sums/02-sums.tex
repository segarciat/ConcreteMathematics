%%%%%%%%%%%%%%%%%%%%%%%%%%%%%%%%%%%%%%%%%%%%%%%%%%%%%%%%%%%%%%%
% Welcome to the MAT320 Homework template on Overleaf -- just edit your
% LaTeX on the left, and we'll compile it for you on the right.
%%%%%%%%%%%%%%%%%%%%%%%%%%%%%%%%%%%%%%%%%%%%%%%%%%%%%%%%%%%%%%%
% --------------------------------------------------------------
% Based on a homework template by Dana Ernst.
% --------------------------------------------------------------
% This is all preamble stuff that you don't have to worry about.
% Head down to where it says "Start here"
% --------------------------------------------------------------

\documentclass[12pt]{article}

\usepackage{graphicx}
\graphicspath{{./images/}}
\usepackage[margin=1in]{geometry} 
\usepackage{amsmath,amsthm,amssymb}
% https://tex.stackexchange.com/questions/146306/how-to-make-horizontal-lists
\usepackage[inline]{enumitem} % allows using letters in enumerate list environment
\usepackage{manfnt} % https://tex.stackexchange.com/questions/40538/symbol-for-cube

% source: https://stackoverflow.com/questions/3175105/inserting-code-in-this-latex-document-with-indentation

\usepackage{listings}
\usepackage{color}

\definecolor{dkgreen}{rgb}{0,0.6,0}
\definecolor{gray}{rgb}{0.5,0.5,0.5}
\definecolor{mauve}{rgb}{0.58,0,0.82}

\lstset{frame=tb,
	language=C, % language for code listing
	aboveskip=3mm,
	belowskip=3mm,
	showstringspaces=false,
	columns=flexible,
	basicstyle={\small\ttfamily},
	numbers=none,
	numberstyle=\tiny\color{gray},
	keywordstyle=\color{blue},
	commentstyle=\color{dkgreen},
	stringstyle=\color{mauve},
	breaklines=true,
	breakatwhitespace=true,
	tabsize=4
}

\newcommand{\N}{\mathbb{N}}
\newcommand{\Z}{\mathbb{Z}}

\newenvironment{ex}[2][Exercise]{\begin{trivlist}
		\item[\hskip \labelsep {\bfseries #1}\hskip \labelsep {\bfseries #2.}]}{\end{trivlist}}

\newenvironment{sol}[1][Solution]{\begin{trivlist}
		\item[\hskip \labelsep {\bfseries #1:}]}{\end{trivlist}}


\begin{document}

% --------------------------------------------------------------
%                         Start here
% --------------------------------------------------------------

\noindent Sergio Garcia Tapia \hfill

\noindent{\small Concrete Mathematics, by Graham, Knuth, and Patashnik} \hfill 

\noindent{\small Chapter 1: Recurrent Problems} \hfill 

\noindent\today
\subsection*{Warmups}
\begin{ex}{1}
	What does the notation
	\[
	\sum_{k=4}^{0}q_k
	\]
	mean?
\end{ex}

\begin{sol}
	It is the sum of all terms $q_k$ whose index  $k$ is an integer between the lower limit of
	$4$ and the upper limit $0$, inclusive. The set of numbers satisfying this property is empty,
	so this is an empty sum with a value fo 0.
\end{sol}

\begin{ex}{2}
	Simplify the expression $x\cdot ([x>0]-[x<0])$.
\end{ex}

\begin{sol}
	Here, $[x>0]$ is the boolean function that is $1$ if $x>0$ and $0$ otherwise; similar for
	$[x<0]$. Suppose that $x>0$. Then the $[x>0]=1$ and $[x<0]=0$, so the expression simplifies
	to $x\cdot (1-0) = x$. If $x<0$, it simplifies to $x\cdot (0-1)=-x$. If $x=0$, then it's just 0.
	Hence
	\[
	x\cdot ([x>0]-[x<0])=|x|.
	\]
\end{sol}

\begin{ex}{3}
	Demonstrate your understanding of $\Sigma$-notation by writing out the sums
	\[
	\sum_{0\leq k \leq 5}a_k
	\quad\text{and}\quad
	\sum_{0\leq k^2\leq 5}a_{k^2}
	\]
	in full. (Watch out --- the second sum is a bit tricky).
\end{ex}

\begin{sol}
	Recalling that $[0\leq k\leq 5]$ is the boolean function that is $1$ for the
	statement
	\[
	P(k): 0\leq k \leq 5
	\]
	\[
	\sum_{0\leq k\leq 5}=a_k=\sum_{k}a_k[0\leq k\leq 5]=a_0+a_1+a_2+a_3+a_4+a_5
	\]
	Similarly, the only integers $k\in\mathbb{Z}$ satisfying $0\leq k^2\leq 5$ are
	$k=-2$, $k=-1$, $k=0$, $k=1$, and $k=2$, so
	\[
	\sum_{0\leq k^2\leq 5}a_{k^2}=a_{(-2)^2}+a_{(-1)^2}+a_{0^2}+a_{1^2}+a_{2^2}=a_2+a_1+a_0+a_1+a_4
	\]
\end{sol}

\begin{ex}{4}
	Express the triple sum
	\[
	\sum_{1\leq i<j<k\leq 4} a_{ijk}
	\]
	as a three-fold summation (with three $\Sigma$'s),
	\begin{enumerate}[label=(\alph*)]
		\item summing first on $k$, then $j$, then $i$;
		\item summing first on $i$, then $j$, then $k$.
	\end{enumerate}
	Also write your triple sums out in full without $\Sigma$-notation, using parentheses to show
	what is being added together first.
\end{ex}

\begin{sol}
	The expanded sum fro the expression given is
	\begin{align*}
		\sum_{1\leq i<j<k\leq 4}a_{ijk}&=a_{123}+a_{124}+a_{134}+a_{234}
	\end{align*}
	\begin{enumerate}[label=(\alph*)]
		\item 	The index condition for the summation can be factored as
		\[
		[1\leq i<j<k\leq 4]=[1\leq i<4][i<j<4][j\leq k\leq 4]
		\]
		Therefore, 
		\begin{align*}
			\sum_{1\leq i<j<k\leq 4} a_{ijk}&=
			\sum_{1\leq i < 4}\sum_{i<j < 4}\sum_{j<k\leq 4}a_{ijk}\\
			&=\left(a_{123}+a_{124}\right)\quad \text{(terms with $i=1, j=2$)}\\
			&+(a_{134})\qquad \text{(terms with $i=1$, $j=3$)}\\
			&+(a_{234}) \qquad\text{(terms with $i=2, j=3$)}
		\end{align*}
		The parentheses show the inner sum $k$ for each fixed $i,j$. Note
		that when $i=3$, we require $i<j$ and $j<k$, which implies $k\geq 5$,
		and no such terms exist.
		\item A different, but equivalent, factorization is
		\[
		[1\leq i<j<k\leq 4]=[1<k\leq 4][1<j<k][1\leq i < j]
		\]
		The sum then becomes
		\begin{align*}
			\sum_{1 < k \leq 4}\sum_{1<j<k}\sum_{1\leq i<j}a_{ijk}
			&=(a_{124})\qquad\text{(terms with $k=4$, $j=2$)}\\
			&+(a_{134}+a_{234})\quad \text{(terms with $k=4$, $j=3$)}\\
			&+(a_{123})\qquad\text{(term with $k=3$)}
		\end{align*}
	\end{enumerate}
\end{sol}

\begin{ex}{5}
	What's wrong with the following derivation?
	\begin{align*}
		\left(\sum_{j=1}^{n}a_j\right)
		\left(\sum_{k=1}^{n}\frac{1}{a_k}\right)
		&=\sum_{j=1}^{n}\sum_{k=1}^{n}\frac{a_j}{a_k}
		&=\sum_{k=1}^{n}\sum_{k=1}^{n}\frac{a_k}{a_k}
		&=\sum_{k=1}^{n}n=n^2
	\end{align*}
\end{ex}

\begin{sol}
	The second equality is incorrect. In writing
	\[
	\sum_{j=1}^{n}\sum_{k=1}^{n}\frac{a_j}{a_k}
	=\sum_{k=1}^{n}\sum_{k=1}^{n}\frac{a_k}{a_k}
	\]
	we are replacing index $j$ with the index $k$ that is already in use,
	introducing ambiguity about two otherwise independent indices.
\end{sol}

\begin{ex}{6}
	What is the value of $\sum_{k}[1\leq j\leq k\leq n]$, as a function of $j$ and $n$?
\end{ex}

\begin{sol}
	Factoring the summand condition, we get
	\begin{align*}
		\sum_{k}[1\leq j\leq k\leq n]&=\sum_{k}[1\leq j\leq n][j\leq k\leq n]\\
		&=[1\leq j\leq n]\sum_{k=j}^{n}\\
		&=[1\leq j\leq n](n-j+1)
	\end{align*}
	because the sum is essentially the count of numbers between $j$ and $n$ for $j\leq n$.
\end{sol}

\begin{ex}{7}
	Let $\nabla f(x)=f(x)-f(x-1)$. What is $\nabla \left(x^{\overline{m}}\right)$?
\end{ex}

\begin{sol}
	Recall that $x^{\overline{m}}$ is ``$x$ to the $m$ rising", or rising power, and it is defined
	as the $m$-term product
	\[
	x^{\overline{m}}=x(x+1)(x+2)\cdots (x+(m-1))
	\]
	Then
	\begin{align*}
		\nabla \left(x^{\overline{m}}\right)&=\left(x^{\overline{m}}\right)-\left(x-1\right)^{\overline{m}}\\
		&=x(x+1)(x+2)\cdots (x+(m-1))-(x-1)(x)(x+1)\cdots (x+(m-2))\\
		&=x(x+1)(x+2)\cdots(x+(m-2))\cdot \left[x+(m-1)-(x-1)\right]\\
		&=mx(x+1)(x+2)\cdots (x+(m-2))\\
		&=m\left(x^{\overline{m-1}}\right)
	\end{align*}
\end{sol}

\begin{ex}{8}
	What is the value of $0^{\underline{m}}$, when $m$ is a given integer?
\end{ex}

\begin{sol}
	It is 0 if $m=0$. If $m>0$, then
	\[
	0^{\underline{m}}=0\cdot (0-1)(0-2)\cdots(0-(m-1))=0
	\]
	If $m<0$, then
	\[
	0^{\underline{m}}=\frac{1}{(0+1)(0+2)\cdots (0+(-m))}=\frac{1}{1^{\overline{m}}}=\frac{1}{m!}
	\]
\end{sol}

\begin{ex}{9}
	What is the law of exponents for rising factorial powers, analogous to (2.52)?
	Use this to define $x^{-\overline{n}}$.
\end{ex}

\begin{sol}
	Note that
	\begin{align*}
		x^{\overline{3}}&=x(x+1)(x+2)\\
		x^{\overline{2}}&=x(x+1)\\
		x^{\overline{1}}&=x\\
		x^{\overline{0}}&=1\\
	\end{align*}
	In other words, we divide by $(x+2)$, then by $(x+1)$ then by $x$. Each time, we divide by
	$x+k$, and we decrease $k$ by $1$. It would seem that to continue the pattern we would want
	to next divide by $(x-1)$, then $(x-2)$, and so on, so that
	\begin{align*}
		x^{\overline{-1}}&=\frac{1}{x-1}\\
		x^{\overline{-2}}&=\frac{1}{(x-1)(x-2)}
	\end{align*}
	and so on. Note that
	\begin{align*}
		x^{\overline{2}}\cdot (x+2)^{\overline{3}}&=x(x+1)\cdot (x+2)(x+3)(x+4)=x^{\overline{m}}
	\end{align*}
	In general,
	\[
	x^{\overline{m+n}}=x^{\overline{m}}(x+m)^{\overline{n}}
	\]
	Now we can use this to define $x^{-n}$ by letting $m=-n$:
	\begin{align*}
		1&=x^{\overline{0}}\\
		&=x^{\overline{-n+n}}\\
		&=x^{\overline{-n}}(x-n)^{\overline{n}}
	\end{align*}
	Hence
	\[
	x^{\overline{-n}}=\frac{1}{(x-n)^{\overline{n}}}=\frac{1}{(x-1)^{\underline{n}}}
	\]
\end{sol}

\begin{ex}{10}
	The text derives the formula for the difference of a product
	\[
	\Delta (uv)=u\Delta v + Ev\Delta u
	\]
	How can this formula be correct, when the left-hand side is symmetric with respect to $u$
	and $v$ but the right-hand side is not? 
\end{ex}

\begin{sol}
	In the derivation of $\Delta (uv)$, we added 0 in the form of $u(x)v(x+1)-u(x)v(x+1)$ to
	simplify the difference. Here, we arbitrarily chose to apply the shift operator to $v$.
	However, we could have equivalently applied it to $u$, to get
	\begin{align*}
		\Delta(uv)&=u(x+1)v(x+1)-u(x)v(x)\\
		&=u(x+1)v(x+1)-u(x+1)v(x)+u(x+1)v(x)-u(x)v(x)\\
		&=Eu\Delta v+ v\Delta u
	\end{align*}
	Therefore, in the text definition, we could certainly switch what the operators are
	being applied to, and obtain an equivalent expression.
\end{sol}
\subsection*{Basics}
\begin{ex}{11}
	The general rule (2.56):
	\[
	\sum u\Delta v=uv-\sum Ev\Delta u
	\]
	 for summation by parts is equivalent to
	 \begin{align*}
	 	\sum_{0\leq k< n}(a_{k+1}-a_k)b_k&=a_nb_n-a_0b_0\\
	 	&-\sum_{0\leq k< n}a_{k+1}(b_{k+1}-b_k),\quad \text{ for } n\geq 0.
	 \end{align*}
	 Prove this formula directly by using the distributive, pairing, and commutative
	 laws.
\end{ex}

\begin{sol}
	\begin{proof}
		\begin{align*}
			\sum_{0\leq k< n}(a_{k+1}-a_k)b_k
			&=\sum_{0\leq k< n}a_{k+1}b_k-\sum_{0\leq k< n}a_{k}b_k\\
			&=\sum_{0\leq k< n}a_{k+1}b_k-\left(a_0b_0-a_nb_n+\sum_{1\leq k\leq n}a_{k}b_k\right)\\
			&=\sum_{0\leq k< n}a_{k+1}b_k-\left(a_0b_0-a_nb_n+\sum_{1\leq k+1 \leq n}a_{k+1}b_{k+1}\right)\\
			&=\sum_{0\leq k< n}a_{k+1}b_k-\left(a_0b_0-a_nb_n+\sum_{0\leq k <n}a_{k+1}b_{k+1}\right)\\
			&=a_nb_n-a_0b_0+\sum_{0\leq k<n}a_{k+1}b_k-\sum_{0\leq k<n}a_{k+1}b_{k+1}\\
			&=a_nb_n-a_0b_0-\sum_{0\leq k<n}(b_{k+1}-b_k)a_{k+1},
		\end{align*}
		where the first equality follows from the pairing law, the second comes from
		splitting off the sum, which is a consequence of the pairing
		law and the commutative law (notice $0\leq k<n$ changed to $1\leq k\leq n$),
		the third from the commutative law because we replaced $k$ by $k+1$, the fourth
		by simplifying the bounds, the fifth from the distributive law, and the last
		by the pairing law.
	\end{proof}
\end{sol}

\begin{ex}{12}
	Show that the function $p(k)=k+(-1)^kc$ is a permutation of the
	set of all integers, whenever $c$ is an integer.
\end{ex}

\begin{sol}
	\begin{proof}
		A permutation of the set of all integers is a $1-1$ correspondence. First, we prove that
		$p$ is 1-1. Suppose that $p(k_1)=p(k_2)$, meaning that
		\[
		k_1+(-1)^{k_1}c=k_2+(-1)^{k_2}c\quad\implies\quad k_2-k_1=(-1)^{k_1}c-(-1)^{k_2}c
		\]
		Suppose that $k_1$ is even and $k_2$ is odd. Then $(-1)^{k_1}=c$ and $(-1)^{k_2}=-$,
		$k_2-k_1=2c$. But this is impossible because the difference of an odd and
		even number is odd. Hence, $k_1$ and $k_2$ are both even, implying that $k_1=k_2$.
		\
		
		Next, we show that $p$ is onto. Suppose $c$ is even. Let $2m$ be any even integer.
		If we let $k=2m-c$, then $k$ is even, so $p(k)=k+(-1)^{k}c=k+c=(2m-c)+c=2m$.
		If $2m+1$ is an odd integer, choose $k=2m+1+c$. Then $k$ is odd, so $(-1)^k$ is -1,
		and hence $p(k)=k+(-1)^cm=k-c=(2m+1+c)-c=2m+1$. Hence, if $c$ is even, then $p$ is
		onto. A similar argument works when $c$ is odd. Hence, $p$ is onto, and is thus a 1-1
		corresponding from $\mathbb{Z}$ onto itself, meaning that it is a permutation.
	\end{proof}
\end{sol}

\begin{ex}{13}
	Use the repertoire method to find a closed form for $\sum_{k=0}^{n}(-1)^kk^2$.
\end{ex}

\begin{sol}
	Let $S_n=\sum_{k=0}^{n}(-1)^kk^2$. To leverage the repertoire method, we express it
	as a recurrence relation, like so:
	\begin{align*}
		S_0&=0;\\
		S_n&=S_{n-1}+(-1)^nn^2.
	\end{align*}
	Suppose $R_n$ is the more general recurrence
	\begin{align*}
		R_0&=\alpha;\\
		R_n&=R_{n-1}+(-1)^n\left(\beta+\gamma n+\lambda n^2\right).
	\end{align*}
	Then $R(n)=A(n)\alpha + B(n)\beta + C(n)\gamma + D(n)\lambda$. To solve for the coefficient
	functions, we choose different solution functions, starting with the simple $R(n)=1$.
	Then the recurrence becomes
	\begin{align*}
		1&=\alpha;\\
		1&=1 + (-1)^n(\beta+\gamma n +\lambda n^2)\quad\iff \quad
		0=(-1)^n(\beta+\gamma n + \lambda n^2)
	\end{align*}
	Hence $\alpha =1 $. In the second equation, the product is only 0 if the parenthesized expression
	is $0$. This immediately implies that $\beta=\gamma=\lambda=0$.  Hence,
	\[
	1=R(n)=A(n)
	\]
	Next, we go with  $R(n)=(-1)^n$,
	so the recurrence becomes:
	\begin{align*}
		1&=\alpha;\\
		(-1)^n&=(-1)^{n-1}+(-1)^n(\beta+\gamma n + \lambda n^2)
		\quad\iff\quad
		2=\beta + \gamma n + \lambda n^2
	\end{align*}
	Hence, $\alpha=1$, $\beta=2$, and $\gamma=\lambda=0$. In this case,
	\begin{align*}
		(-1)^n=R(n)=A(n)+2B(n)
	\end{align*}
	Recalling that $A(n)=1$, it follows that
	\[
	B(n)=\frac{1}{2}(-1)^n-\frac{1}{2}
	\]
	Next, we consider $R(n)=(-1)^nn$. In this case, the recurrence becomes:
	\begin{align*}
		0&=\alpha;\\
		(-1)^nn&=(-1)^{n-1}(n-1)+(-1)^n(\beta + \gamma n + \lambda n^2)
		\quad\iff\quad
		2n-1=\beta + \gamma n + \lambda n^2
	\end{align*}
	Hence, $\alpha=0$, $\beta=-1$, $\gamma=2$, and $\lambda= 0$. This implies that
	\begin{align*}
		(-1)^nn=R(n)=-B(n)+2C(n)
	\end{align*}
	Since we have already determined $B(n)$, we can solve for $C(n)$:
	\begin{align*}
		C(n)=\frac{1}{2}(-1)^n n + \frac{1}{4}(-1)^n-\frac{1}{4}
	\end{align*}
	We need one last equation to determine $D(n)$. Let $R(n)=(-1)n^2$. The recurrence becomes:
	\begin{align*}
		0&=\alpha;\\
		(-1)^nn^2&=(-1)^{n-1}(n-1)^2+(-1)^n(\beta+\gamma n+ \lambda n^2)
	\end{align*}
	We can simply the second equation by dividing by $(-1)^n$, expanding $(n-1)n^2$,
	and simplifying:
	\begin{align*}
		n^2&=-(n-1)^2+(\beta +\gamma n + \lambda n^2)\\
		n^2&=(-1+2n-n^2)+(\beta +\gamma n + \lambda n^2)\\
		2n^2-2n+1&=\beta+\gamma n + \lambda n^2
	\end{align*}
	Equating coefficients, we conclude that $\alpha=0$, $\beta=1$, $\gamma = -2$, and
	$\lambda = 2$. Hence
	\begin{align*}
		(-1)^nn^2=R(n)=B(n)-2C(n)+2D(n).
	\end{align*}
	Substituting our solutions for $B(n)$ and $C(n)$, we get
	\begin{align*}
		D(n)&=\frac{1}{2}(-1)^n n^2 + \frac{1}{2}(-1)^n n + \frac{1}{4}(-1)^n - \frac{1}{4}
		-\frac{1}{4}(-1)^n + \frac{1}{4}\\
		&=(-1)^n\frac{(n^2 + n)}{2}.
	\end{align*}
	Going back to our series $S_n=\sum_{k=0}^{n}(-1)^k k^2$, recall that its corresponding
	recurrence is
	\begin{align*}
		S_0&=0;\\
		S_n&=S_{n-1}+(-1)^nn^2.
	\end{align*}
	This means $\alpha=0$, $\beta=0$, $\gamma=0$, and $\lambda=1$. Hence, $S_n=S(n)=D(n)$, so
	\begin{align*}
		\sum_{k=0}^{n}(-1)^k k^2
		=(-1)^n\frac{(n^2 + n)}{2}
	\end{align*}
\end{sol}

\begin{ex}{14}
	Evaluate $\sum_{k=1}^{n} k 2^k$ by rewriting it as the multiple sum
	$\sum_{1\leq j \leq k \leq n} 2^k$.
\end{ex}

\begin{sol}
	\
	 Recall that
	 \[
	 [1\leq j\leq n][j\leq k\leq n]=[1\leq j \leq k \leq n]=[1\leq k \leq n] [1\leq j \leq k].
	 \]
	 We can use this to re-write the sum
	 \begin{align*}
	 	\sum_{1\leq k\leq n}k2^k&=\sum_{1\leq k\leq n}k2^k\cdot \frac{1}{k}\sum_{1\leq j\leq k}1\\
	 	&=\sum_{1\leq k\leq n}\sum_{1\leq j\leq k}k2^k\cdot \frac{1}{k}\\
	 	&=\sum_{1\leq j\leq k\leq n}2^k\\
	 	&=\sum_{1\leq j\leq n}\sum_{j\leq k\leq n}2^k\\
	 	&=\sum_{1\leq j\leq n}\sum_{j\leq k + j\leq n}2^{k+j}\\
	 	&=\sum_{1\leq j\leq n}2^{j}\sum_{0\leq k\leq n-j}2^k\\
	 	&=\sum_{1\leq j\leq n}2^{j}(2^{n-j+1}-1)\\
	 	&=\sum_{1\leq j\leq n}2^{n+1}-\sum_{1\leq j\leq n}2^{j}\\
	 	&=n\cdot 2^{n+1}- \sum_{0\leq j+1\leq n}2^{j+1}\\
	 	&=n\cdot 2^{n+1}- \sum_{0\leq j\leq n-1}2^{j+1}\\
	 	&=n\cdot 2^{n+1}-2^{n+1}+2
	 \end{align*}
\end{sol}

\begin{ex}{15}
	Evaluate $\text{\mancube}_n=\sum_{k=1}^{n}k^3$ by using the text's Method 5 (Expand and Contract)
	as follows: First write $\text{\mancube}_n+\square_n=2\sum_{1\leq j\leq k\leq n}jk$; then apply
	Equation~\ref{eq:2.33}, which is (2.33) in the book, given below:
	\begin{equation}\label{eq:2.33}
			\sum_{1\leq j\leq k\leq n}a_ja_k=\frac{1}{2}\left(
		\left(\sum_{k=1}^{n}a_k\right)^2
		+\sum_{k=1}^{n}a_k^2
		\right)
	\end{equation}
\end{ex}

\begin{sol}
	Recall that
	\[
	\sum_{1\leq j\leq k}j=\frac{k(k+1)}{2}
	\]
	We use this equation below:
	\begin{align*}
		\text{\mancube}_n+\square_n
		&=\sum_{k=1}^{n}k^3+\sum_{k=1}^{n}k^2\\
		&=\sum_{1\leq k\leq n}k^2(k+1)\\
		&=\sum_{1\leq k\leq n}k^2(k+1)
		\cdot
		\frac{2}{(k+1)k} \sum_{1\leq j\leq k}j\\
		&=2\sum_{1\leq k\leq n}\sum_{1\leq j\leq k}jk\\
		&=2\sum_{1\leq j\leq k\leq n}jk\\
	\end{align*}
	Now applying Equation~\ref{eq:2.33} (2.33 in book) with $a_j=j$ and $a_k=k$:
	\begin{align*}
		\text{\mancube}_n+\square_n&=
		2\sum_{1\leq j\leq k\leq n}jk\\
		&=\left(
		\left(\sum_{k=1}^{n}k\right)^2
		+\sum_{k=1}^{n}k^2
		\right)\\
		&=\left(
		\left(\frac{n(n+1)}{2}\right)^2
		+\frac{(n+1)(n+\frac{1}{2})n}{3}
		\right)\\
		&=\left(
		\left(\frac{n(n+1)}{2}\right)^2
		+\square_n
		\right)
	\end{align*}
	Subtracting $\square_n$ on both sides results in
	\begin{align*}
		\text{\mancube}_n=\left(\frac{n(n+1)}{2}\right)^2
	\end{align*}
\end{sol}

\begin{ex}{16}
	Prove that $x^{\underline{m}}/(x-n)^{\underline{m}}=x^{\underline{n}}/(x-m)^{\underline{n}}$,
	unless one of the denominators is zero.
\end{ex}

\begin{sol}
	\begin{proof}
		If $m=n$, then the statement is trivially true. Suppose, without loss generality,
		that $m>n$. Then
		\begin{align*}
			\frac{x^{\underline{m}}}{(x-n)^{\underline{m}}}
			&=\frac{x(x-1)\cdots(x-(n-1))(x-n)(x-(m-1))}
			{(x-n)(x-n-1))\cdots (x-(m-1))(x-m)\cdots  (x-n-(m-1))}\\
			&=\frac{x(x-1)\cdots(x-(n-1))}{(x-m)\cdots (x-m-1)\cdots (x-n-(m-1))}\\
			&=\frac{x^{\underline{n}}}{(x-m)^{\underline{n}}}
		\end{align*}
	\end{proof}
\end{sol}

\begin{ex}{17}
	Show that the following formulas can be used to convert between rising and falling
	factorial powers, for all integers $m$:
	\begin{align*}
		x^{\overline{m}}&=(-1)^m(-x)^{\underline{m}}=(x+m-1)^{\underline{m}}=1/(x-1)^{\underline{-m}};\\
		x^{\underline{m}}&=(-1)^m(-x)^{\overline{m}}=(x-m+1)^{\overline{m}}=1/(x+1)^{\overline{-m}}.\\
	\end{align*}
\end{ex}

\begin{sol}
	Starting with $x^{\overline{m}}=1/(x-1)^{\underline{-m}}$:
	\begin{align*}
		\frac{1}{(x-1)^{\underline{-m}}}&=\frac{1}{\frac{1}{(x-1+1)(x-1+2)\cdots(x-1+m)}}\\
		&=x(x+1)(x+2)\cdots(x+(m-1))\\
		&=x^{\overline{m}}
	\end{align*}
	We can prove $x^{\overline{m}}=(x+m-1)^{\underline{m}}$ by reversing the order
	of the product:
	\begin{align*}
		x^{\overline{m}}=x(x+1)(x+2)\cdots (x+(m-2))(x+(m-1))
		&=(x+(m-1))(x+(m-1)-1)\cdots (x+2)(x+1)x\\
		&=(x+m-1)^{\underline{m}}
	\end{align*}
	Lastly, we can prove $x^{\overline{m}}=(-1)^m(-x)^{\underline{m}}$ by 
	multiplying by $(-1)$ and changing the additions to subtractions:
	\begin{align*}
		x^{\overline{m}}&=x(x+1)(x+2)\cdots(x+(m-1))\\
		&=[(-1)\cdot (-x)][(-1)\cdot (-x-1)]\cdot [(-1)\cdot (-x-2)]\cdots [(-1)[-x-(m-1)]]\\
		&=(-1)^m(-x)(-x-1)(-x-2)\cdots(-x-(m-1))\\
		&=(-1)^m(-x)^{\underline{m}}.
	\end{align*}
	We could use the equations we just proved to prove the $x^{\underline{m}}$ equations,
	but the same arguments work. First, we can prove $x^{\underline{m}}=(-1)^m(-x)^{\overline{m}}$
	by multiplying every term by $(-1)$ and reversing the order of the subtraction:
	\begin{align*}
		x^{\underline{m}}&=x(x-1)(x-2)\cdots(x-(m-1))\\
		&[(-1)\cdot (-x)]\cdot [(-1)\cdot (-x+1)]\cdot [(-1)(-x+2)]\cdots[(-1)(-x+m-1)]\\
		&=(-1)^n(-x)(-x+1)\cdots (-x+(m-1))\\
		&=(-1)^n(-x)^{\overline{m}}
	\end{align*}
	Next, we can prove $x^{\underline{m}}=(x-m+1)^{\overline{m}}$ by writing the product
	in reverse:
	\begin{align*}
		x^{\underline{m}}&=x(x-1)(x-2)\cdots (x-(m-2))(x-(m-1))\\
		&=(x-(m-1))(x-(m-2))\cdots (x-2)(x-1)\\
		&=(x-m+1)(x-m+2)\cdots (x-m+1 +(m-2))(x-m+1+(m-1))\\
		&=(x-m+1)(x-m+2)\cdots (x-1)x\\
		&=(x-m+1)^{\underline{m}}
	\end{align*}
	For the last falling factorial equation, we start from the right-side. From Exercise 9,
	it follows that
	\[x^{\overline{-m}}=\frac{1}{(x-m)^{\overline{m}}}=\frac{1}{(x-1)^{\underline{m}}}\]
	Therefore, if we start from the right, replace $x$ with $x+1$, and use the reciprocal:
	\[
	x^{\underline{m}}=((x+1)-1)^{\underline{m}}=\frac{1}{(x+1)^{\overline{-m}}}
	\]
\end{sol}

\begin{ex}{18}
	Let $\Re z$ and $\Im z$ be the real and imaginary parts of the complex number $z$.
	The absolute value $|z|$ is $\sqrt{(\Re z)^2+(\Im z)^2}$. A sum $\sum_{k\in K} a_k$
	of complex terms $ak$ is said to converge absolutely when the real-valued sums
	$\sum_{k\in K}\Re a_k$ and $\sum_{k\in K} \Im a_k$ both converge absolutely.
	Prove that $\sum_{k\in K} a_k$ converges absolutely if and only if there is a
	boundary constant $B$ such that $\sum_{k\in F}|a_k|\leq B$ for all finite subsets
	$F \subseteq K$.
\end{ex}

\begin{sol}
	\begin{proof}
		(``$\implies$"): Suppose that $\sum_{k\in K}\Re a_z$ and $\sum_{k\in K} \Im a_z$
		both converge absolutely. Recall that
		\[
		\Re a_k=\Re a_k^+-\Re a_k^-,\quad \Im a_k=\Im a_k^+-\Im a_k^-,
		\]
		where the operands on the right-hand side are all non-negative.
		Then the absolute convergence of $\Re a_k$ and $\Im a_k$ imply that the
		sums
		\begin{align*}
			\sum_{k\in K}\Re a_k^+,\quad 
			\sum_{k\in K}\Re a_k^-,\quad 
			\sum_{k\in K}\Im a_k^+,\quad 
			\sum_{k\in K}\Im a_k^-,
		\end{align*}
		all converge.  Therefore, there are constants $B_1,B_2,B_3,B_4\in \mathbb{R}$ such that
		for all finite subsets $F\subseteq K$, we have
		\begin{align*}
			\sum_{k\in F}\Re a_k^+\leq \frac{B_1}{4},\quad 
			\sum_{k\in F}\Re a_k^-\leq \frac{B_2}{4},\quad 
			\sum_{k\in F}\Im a_k^+\leq \frac{B_3}{4},\quad 
			\sum_{k\in F}\Im a_k^-\leq \frac{B_4}{4},\quad.
		\end{align*}
		Pick $B=\max\{B_1,B_2,B_3,B_4\}$. Then, by the triangle inequality,
		Recalling that the triangle inequality says that $|x+y|\leq |x|+|y|$, we have
		\begin{align*}
			\sum_{k\in F}|a_k|&=\sum_{k\in F}|\Re a_k+i\Im a_k|\\
			&\leq \sum_{k\in F}(|\Re a_k| + |\Im a_k|)\\
			&=\sum_{k\in F}|\Re a_k| + \sum_{k\in F}|\Im a_k|\\
			&=\sum_{k\in F}|\Re a_k^+ - \Re a_k^-| + \sum_{k\in F}|\Im a_k^+ - \Im a_k^-|\\
			&\leq \sum_{k\in F}|\Re a_k^+|+\sum_{k\in F} |\Re a_k^-|
			+ \sum_{k\in F}|\Im a_k^+| + \sum_{k\in F} |\Im a_k^-|\\
			&\leq \frac{B_1}{4}+\frac{B_2}{4}+\frac{B_3}{4}+\frac{B_4}{4}\\
			&\leq \frac{B}{4}+\frac{B}{4}+\frac{B}{4}+\frac{B}{4}\\
			&= B
		\end{align*}
		This concludes the proof of the forward direction.
		
		\
		
		\noindent (``$\impliedby$"):  Suppose now that there is $B\in\mathbb{R}$
		such that $\sum_{k\in F}|a_k|\leq B$ for all finite subsets $F\subseteq K$.
		Note that $\Re a_k\leq |\Re a_k|\leq |a_k|$ and $\Im a_k\leq |\Im a_k|\leq |a_k|$.
		Therefore, it follows that
		\begin{align*}
			\sum_{k\in F}\Re a_k\leq \sum_{k\in F}|\Re a_k|\leq \sum_{k\in F}|a_k|\leq B
			\quad\text{and}\quad
			\sum_{k\in F}\Im a_k\leq \sum_{k\in F}|\Im a_k|\leq \sum_{k\in F}|a_k|\leq B.
		\end{align*}
		which means that $\sum_{k\in F}\Re a_k$ and $\sum_{k\in F}\Im a_k$ both
		converge absolutely.
	\end{proof}
\end{sol}
\subsection*{Homework}

\end{document}